\documentclass[11pt]{article}
\input{headers3}

\usepackage{fancyhdr}   
\pagestyle{fancy}      
\lhead{MA453 Spring 2018 - Homework 3}               
\rhead{Harris Christiansen (christih@purdue.edu)}

\usepackage{mathrsfs}
\usepackage{amsmath}
\usepackage[strict]{changepage}  
\newcommand{\nextoddpage}{\checkoddpage\ifoddpage{\ \newpage\ \newpage}\else{\ \newpage}\fi}

\begin{document}

\title{Homework 3}
\date{p63 D.1, p64 E.5 and E.6, p65 G.1 and G.2, p75 A.1, A.2, and A.5, p76 B.2}
\maketitle

\thispagestyle{fancy}  
\pagestyle{fancy}      

\begin{enumerate}

%%% Problem p63 D.1
\item {\bfseries p63 D.1.} Find the composite function, $f \circ g$ and $g \circ f$:

	$f : \mathbb{R} \rightarrow \mathbb{R}$ is defined by $f(x) = \sin(x)$ \\
	$g : \mathbb{R} \rightarrow \mathbb{R}$ is defined by $g(x) = e^x$
  
	{\bfseries Solution.}

%%% Problem p64 E.5
\item {\bfseries p64 E.5.} $f$ is a bijective function. Describe its inverse.

	$A = \{a,b,c,d\}, B = \{1,2,3,4\}$ and $f : A \rightarrow B$ is given by: \\
	$\begin{pmatrix}
		a \hspace{6mm} b \hspace{6mm} c \hspace{6mm} d \\
		3 \hspace{6mm} 1 \hspace{6mm} 2 \hspace{6mm} 4 \\
	\end{pmatrix}$
  
	{\bfseries Solution.}

%%% Problem p64 E.6
\item {\bfseries p64 E.6.} $f$ is a bijective function. Describe its inverse.

	$G$ is a group, $a \in G$, and $f : G \rightarrow G$ is defined by $f(x) = ax$.
  
	{\bfseries Solution.}
  
\newpage

%%% Problem p65 G.1
\item {\bfseries p65 G.1.} Let $A$, $B$, and $C$ by sets. Prove that if $g \circ f$ is injective, then $f$ is injective.
  
	{\bfseries Solution.}

%%% Problem p65 G.2
\item {\bfseries p65 G.2.} Let $A$, $B$, and $C$ by sets. Prove that if $g \circ f$ is surjective, then $g$ is surjective.
  
	{\bfseries Solution.}
  
\newpage

%%% Problem p75 A.1
\item {\bfseries p75 A.1.} Consider the following permutations $f$, $g$, and $h$ in $S_6$

	$f = \begin{pmatrix}
		1 \hspace{6mm} 2 \hspace{6mm} 3 \hspace{6mm} 4 \hspace{6mm} 5 \hspace{6mm} 6 \\
		6 \hspace{6mm} 1 \hspace{6mm} 3 \hspace{6mm} 5 \hspace{6mm} 4 \hspace{6mm} 2
	\end{pmatrix}$
	$g = \begin{pmatrix}
		1 \hspace{6mm} 2 \hspace{6mm} 3 \hspace{6mm} 4 \hspace{6mm} 5 \hspace{6mm} 6 \\
		2 \hspace{6mm} 3 \hspace{6mm} 1 \hspace{6mm} 6 \hspace{6mm} 5 \hspace{6mm} 4
	\end{pmatrix}$ \\
	$h = \begin{pmatrix}
		1 \hspace{6mm} 2 \hspace{6mm} 3 \hspace{6mm} 4 \hspace{6mm} 5 \hspace{6mm} 6 \\
		3 \hspace{6mm} 1 \hspace{6mm} 6 \hspace{6mm} 4 \hspace{6mm} 5 \hspace{6mm} 2
	\end{pmatrix}$
  
	{\bfseries Solution.} Compute the following:
	
	$f^{-1} = \begin{pmatrix}
		1 \hspace{6mm} 2 \hspace{6mm} 3 \hspace{6mm} 4 \hspace{6mm} 5 \hspace{6mm} 6 \\
		\_ \hspace{6mm} \_ \hspace{6mm} \_ \hspace{6mm} \_ \hspace{6mm} \_ \hspace{6mm} \_
	\end{pmatrix}$
	$g^{-1} = \begin{pmatrix}
		1 \hspace{6mm} 2 \hspace{6mm} 3 \hspace{6mm} 4 \hspace{6mm} 5 \hspace{6mm} 6 \\
		\_ \hspace{6mm} \_ \hspace{6mm} \_ \hspace{6mm} \_ \hspace{6mm} \_ \hspace{6mm} \_
	\end{pmatrix}$ \\
	$h^{-1} = \begin{pmatrix}
		1 \hspace{6mm} 2 \hspace{6mm} 3 \hspace{6mm} 4 \hspace{6mm} 5 \hspace{6mm} 6 \\
		\_ \hspace{6mm} \_ \hspace{6mm} \_ \hspace{6mm} \_ \hspace{6mm} \_ \hspace{6mm} \_
	\end{pmatrix}$ \\
	$f \circ g = \begin{pmatrix}
		1 \hspace{6mm} 2 \hspace{6mm} 3 \hspace{6mm} 4 \hspace{6mm} 5 \hspace{6mm} 6 \\
		\_ \hspace{6mm} \_ \hspace{6mm} \_ \hspace{6mm} \_ \hspace{6mm} \_ \hspace{6mm} \_
	\end{pmatrix}$
	$g \circ f = \begin{pmatrix}
		1 \hspace{6mm} 2 \hspace{6mm} 3 \hspace{6mm} 4 \hspace{6mm} 5 \hspace{6mm} 6 \\
		\_ \hspace{6mm} \_ \hspace{6mm} \_ \hspace{6mm} \_ \hspace{6mm} \_ \hspace{6mm} \_
	\end{pmatrix}$

%%% Problem p75 A.2
\item {\bfseries p75 A.2.} Given p75 A.1, compute the following:

	$f \circ (g \circ h) = $
  
	{\bfseries Solution.}
	
	$g \circ h = \begin{pmatrix}
		1 \hspace{6mm} 2 \hspace{6mm} 3 \hspace{6mm} 4 \hspace{6mm} 5 \hspace{6mm} 6 \\
		\_ \hspace{6mm} \_ \hspace{6mm} \_ \hspace{6mm} \_ \hspace{6mm} \_ \hspace{6mm} \_
	\end{pmatrix}$

%%% Problem p75 A.5
\item {\bfseries p75 A.5.} Given p75 A.1, compute the following:

	$g \circ g \circ g = $
  
	{\bfseries Solution.}
	
	$g \circ g = \begin{pmatrix}
		1 \hspace{6mm} 2 \hspace{6mm} 3 \hspace{6mm} 4 \hspace{6mm} 5 \hspace{6mm} 6 \\
		\_ \hspace{6mm} \_ \hspace{6mm} \_ \hspace{6mm} \_ \hspace{6mm} \_ \hspace{6mm} \_
	\end{pmatrix}$
  
\newpage

%%% Problem p76 B.2
\item {\bfseries p76 B.2.} List the elements of the cyclic subgroup of $S_6$ generated by:

	$f = \begin{pmatrix}
		1 \hspace{6mm} 2 \hspace{6mm} 3 \hspace{6mm} 4 \hspace{6mm} 5 \hspace{6mm} 6 \\
		2 \hspace{6mm} 3 \hspace{6mm} 4 \hspace{6mm} 1 \hspace{6mm} 6 \hspace{6mm} 5
	\end{pmatrix}$
  
	{\bfseries Solution.}

\end{enumerate}

\end{document}
