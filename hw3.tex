\documentclass[11pt]{article}
\input{headers3}

\usepackage{fancyhdr}   
\pagestyle{fancy}      
\lhead{MA453 Spring 2018 - Homework 3}               
\rhead{Harris Christiansen (christih@purdue.edu)}

\usepackage{mathrsfs}
\usepackage{amsmath}
\usepackage[strict]{changepage}  
\newcommand{\nextoddpage}{\checkoddpage\ifoddpage{\ \newpage\ \newpage}\else{\ \newpage}\fi}

\begin{document}

\title{Homework 3}
\date{p63 D.1, p64 E.5 and E.6, p65 G.1 and G.2, p75 A.1, A.2, and A.5, p76 B.2}
\maketitle

\thispagestyle{fancy}  
\pagestyle{fancy}      

\begin{enumerate}

%%% Problem p63 D.1
\item {\bfseries p63 D.1.} Find the composite function, $f \circ g$ and $g \circ f$:

	$f : \mathbb{R} \rightarrow \mathbb{R}$ is defined by $f(x) = \sin(x)$ \\
	$g : \mathbb{R} \rightarrow \mathbb{R}$ is defined by $g(x) = e^x$
  
	{\bfseries Solution.} Let $f, g$ be defined as above, and $x \in \mathbb{R}$ be any value.
	
	$f \circ g$
	\begin{align*}
		f \circ g (x) &= f(g(x)) \\
		&= f(e^x) \\
		&= \sin(e^x) \\
	\end{align*}
	
	$g \circ f$
	\begin{align*}
		g \circ f (x) &= g(f(x)) \\
		&= g(\sin(x)) \\
		&= e^{\sin(x)}
	\end{align*}

%%% Problem p64 E.5
\item {\bfseries p64 E.5.} $f$ is a bijective function. Describe its inverse.

	$A = \{a,b,c,d\}, B = \{1,2,3,4\}$ and $f : A \rightarrow B$ is given by: \\
	$\begin{pmatrix}
		a \hspace{6mm} b \hspace{6mm} c \hspace{6mm} d \\
		3 \hspace{6mm} 1 \hspace{6mm} 2 \hspace{6mm} 4 \\
	\end{pmatrix}$
  
	{\bfseries Solution.} The inverse matrix $f^{-1}: B \rightarrow A$ is defined as:
	
	$\begin{pmatrix}
		1 \hspace{6mm} 2 \hspace{6mm} 3 \hspace{6mm} 4 \\
		b \hspace{6mm} c \hspace{6mm} a \hspace{6mm} d \\
	\end{pmatrix}$
	
	Since this matrix satisfies the property $\forall x \in B, f \circ f^{-1} (x) = x$, we can verify that it is the correct inverse of $f$.

%%% Problem p64 E.6
\item {\bfseries p64 E.6.} $f$ is a bijective function. Describe its inverse.

	$G$ is a group, $a \in G$, and $f : G \rightarrow G$ is defined by $f(x) = ax$.
  
	{\bfseries Solution.} The inverse $f^{-1}: G \rightarrow G$ is defined as:
	
	$$f^{-1}(x) = a^{-1}x$$
	
	where $a^{-1} \in G$ such that $a^{-1}a = aa^{-1} = e$.
	
	This can be verified by the composition: $\forall x \in G, f \circ f^{-1} (x) = x$
	\begin{align*}
		f \circ f^{-1} (x) &= f(f^{-1}(x)) \\
		&= f(a^{-1}x) \\
		&= aa^{-1}x \\
		&= ex \\
		&= x
	\end{align*} \qed
  
\newpage

%%% Problem p65 G.1
\item {\bfseries p65 G.1.} Let $A$, $B$, and $C$ by sets and let $f: A \rightarrow B$ and $g: B \rightarrow C$ be functions. Prove that if $g \circ f$ is injective, then $f$ is injective.
  
	{\bfseries Solution.} Let $f, g$ be defined as above.
	
	\begin{proof} 
		Assume, for the sake of contradiction, that $f$ is not injective. \\
		Given that $f$ is not injective, this implies there exists some $\alpha_1, \alpha_2 \in A$ and $\beta \in B$ such that $f(\alpha_1) = f(\alpha_2) = \beta$. \\
		However, since the composition $g \circ f$ is given to be injective, we know for any $\alpha_1, \alpha_2 \in A$ that $g \circ f (\alpha_{1}) \ne g \circ f (\alpha_{2})$. \\
		Since $f(\alpha_1) = f(\alpha_2) = \beta$, this implies a contradiction that $g \circ f (\beta) \neq g \circ f (\beta)$, therefor $f$ must be injective.
	\end{proof}

%%% Problem p65 G.2
\item {\bfseries p65 G.2.} Let $A$, $B$, and $C$ by sets and let $f: A \rightarrow B$ and $g: B \rightarrow C$ be functions. Prove that if $g \circ f$ is surjective, then $g$ is surjective.
  
	{\bfseries Solution.} Let $f, g$ be defined as above.
  	
  	\begin{proof}
  		Assume, for the sake of contradiction, that $g$ is not surjective. \\
		Given that $g$ is not surjective, this implies there exists some $\zeta \in C$ such that $g(\beta) \neq \zeta$ for any $\beta \in B$. \\ 
		Because $g \circ f$ is surjective, we know for all $\zeta \in C, \exists \alpha \in A$ such that $g \circ f (\alpha) = \zeta$. \\
		Unfolding this statement, it implies $\exists \beta \in B, \alpha \in A$ such that $g(f(\alpha)) = g(\beta) = \zeta$, which is a contradiction to our original assumption. Therefor $g$ must be surjective.
	\end{proof}
  
\newpage

%%% Problem p75 A.1
\item {\bfseries p75 A.1.} Consider the following permutations $f$, $g$, and $h$ in $S_6$

	$f = \begin{pmatrix}
		1 \hspace{6mm} 2 \hspace{6mm} 3 \hspace{6mm} 4 \hspace{6mm} 5 \hspace{6mm} 6 \\
		6 \hspace{6mm} 1 \hspace{6mm} 3 \hspace{6mm} 5 \hspace{6mm} 4 \hspace{6mm} 2
	\end{pmatrix}$
	$g = \begin{pmatrix}
		1 \hspace{6mm} 2 \hspace{6mm} 3 \hspace{6mm} 4 \hspace{6mm} 5 \hspace{6mm} 6 \\
		2 \hspace{6mm} 3 \hspace{6mm} 1 \hspace{6mm} 6 \hspace{6mm} 5 \hspace{6mm} 4
	\end{pmatrix}$ \\
	$h = \begin{pmatrix}
		1 \hspace{6mm} 2 \hspace{6mm} 3 \hspace{6mm} 4 \hspace{6mm} 5 \hspace{6mm} 6 \\
		3 \hspace{6mm} 1 \hspace{6mm} 6 \hspace{6mm} 4 \hspace{6mm} 5 \hspace{6mm} 2
	\end{pmatrix}$
  
	{\bfseries Solution.} Compute the following:
	
	$f^{-1} = \begin{pmatrix}
		1 \hspace{6mm} 2 \hspace{6mm} 3 \hspace{6mm} 4 \hspace{6mm} 5 \hspace{6mm} 6 \\
		\_ \hspace{6mm} \_ \hspace{6mm} \_ \hspace{6mm} \_ \hspace{6mm} \_ \hspace{6mm} \_
	\end{pmatrix}$
	$g^{-1} = \begin{pmatrix}
		1 \hspace{6mm} 2 \hspace{6mm} 3 \hspace{6mm} 4 \hspace{6mm} 5 \hspace{6mm} 6 \\
		\_ \hspace{6mm} \_ \hspace{6mm} \_ \hspace{6mm} \_ \hspace{6mm} \_ \hspace{6mm} \_
	\end{pmatrix}$ \\
	$h^{-1} = \begin{pmatrix}
		1 \hspace{6mm} 2 \hspace{6mm} 3 \hspace{6mm} 4 \hspace{6mm} 5 \hspace{6mm} 6 \\
		\_ \hspace{6mm} \_ \hspace{6mm} \_ \hspace{6mm} \_ \hspace{6mm} \_ \hspace{6mm} \_
	\end{pmatrix}$ \\
	$f \circ g = \begin{pmatrix}
		1 \hspace{6mm} 2 \hspace{6mm} 3 \hspace{6mm} 4 \hspace{6mm} 5 \hspace{6mm} 6 \\
		\_ \hspace{6mm} \_ \hspace{6mm} \_ \hspace{6mm} \_ \hspace{6mm} \_ \hspace{6mm} \_
	\end{pmatrix}$
	$g \circ f = \begin{pmatrix}
		1 \hspace{6mm} 2 \hspace{6mm} 3 \hspace{6mm} 4 \hspace{6mm} 5 \hspace{6mm} 6 \\
		\_ \hspace{6mm} \_ \hspace{6mm} \_ \hspace{6mm} \_ \hspace{6mm} \_ \hspace{6mm} \_
	\end{pmatrix}$

%%% Problem p75 A.2
\item {\bfseries p75 A.2.} Given p75 A.1, compute the following:

	$f \circ (g \circ h) = $
  
	{\bfseries Solution.}
	
	$g \circ h = \begin{pmatrix}
		1 \hspace{6mm} 2 \hspace{6mm} 3 \hspace{6mm} 4 \hspace{6mm} 5 \hspace{6mm} 6 \\
		\_ \hspace{6mm} \_ \hspace{6mm} \_ \hspace{6mm} \_ \hspace{6mm} \_ \hspace{6mm} \_
	\end{pmatrix}$

%%% Problem p75 A.5
\item {\bfseries p75 A.5.} Given p75 A.1, compute the following:

	$g \circ g \circ g = $
  
	{\bfseries Solution.}
	
	$g \circ g = \begin{pmatrix}
		1 \hspace{6mm} 2 \hspace{6mm} 3 \hspace{6mm} 4 \hspace{6mm} 5 \hspace{6mm} 6 \\
		\_ \hspace{6mm} \_ \hspace{6mm} \_ \hspace{6mm} \_ \hspace{6mm} \_ \hspace{6mm} \_
	\end{pmatrix}$
  
\newpage

%%% Problem p76 B.2
\item {\bfseries p76 B.2.} List the elements of the cyclic subgroup of $S_6$ generated by:

	$f = \begin{pmatrix}
		1 \hspace{6mm} 2 \hspace{6mm} 3 \hspace{6mm} 4 \hspace{6mm} 5 \hspace{6mm} 6 \\
		2 \hspace{6mm} 3 \hspace{6mm} 4 \hspace{6mm} 1 \hspace{6mm} 6 \hspace{6mm} 5
	\end{pmatrix}$
  
	{\bfseries Solution.}

\end{enumerate}

\end{document}
